 \documentclass[12pt, oneside]{amsart}   	% use "amsart" instead of "article" for AMSLaTeX format
 \usepackage{geometry}                		% See geometry.pdf to learn the layout options. There are lots.
 \geometry{letterpaper}                   		% ... or a4paper or a5paper or ... 
 %\geometry{landscape}                		% Activate for for rotated page geometry
 \usepackage[parfill]{parskip}    		% Activate to begin paragraphs with an empty line rather than an indent
 \usepackage{graphicx}				% Use pdf, png, jpg, or eps§ with pdflatex; use eps in DVI mode
 								% TeX will automatically convert eps --> pdf in pdflatex	 
 \usepackage{amssymb}
 \usepackage{url}
\usepackage{setspace}
\doublespacing 
 \title{Mental Model Topic Study}
 \author{Abdul Alzaid}
 %\date{}							% Activate to display a given date or no date
 \begin{document}
 \maketitle
 \section{Abstract} % JD: LaTeX has an \abstract{} directive specifically for this purpose.
This paper is introducing the idea of the new technology called "Second Screen" and discussing the ideas, and future potentials of this major technology.  Also, this paper is taking in consideration some mental model guidelines while discussing the Second Screen technology.  Moreover, bringing to the readers some quotes by the broadcasters, relating to the future usage and implementation of this new technology into their broadcast world.  As well as, what makes this technology grows rapidly with the major media companies, to the opposite of that which, to what might affect the spread of this technology at this time.

 \section{Introduction}
In this paper we will be discussing, the definition, history, and the latest offerings the new technology that is called Second Screen has come up with. Also, we will discuss how this new technology relate to guidelines and principles of interaction design. Which, with the results of some recent studies, we will conclude if Second Screen technology might, or might not make the current way of interacting with visual media take a new turn to be widely used, and more interactive, and enjoyable by the users in the future. 

% JD: Also, don't forget that LaTeX quotes should be done as `` and ''. (problem fixed)
However, what is Second Screen?  First before answering that question let us bring back its history back into discussion.  Second screen technology is a derivative of  what's called social television. Moreover, a ``social television is a general term for technology that supports communication and social interaction in either the context of watching television, or related to TV content. It also includes the study of television-related social behavior, devices and networks.``\cite{defOfSocialTV}   However, with the study of television-related social behavior, there had to come new services, applications and changes in content, and device productions. Then, came what we call now second screen technology.  Which when defined we say ``a second screen is a second electronic device used by television viewers to connect to a program they're watching. A second screen is often a smartphone or tablet, where a special complementary app may allow the viewer to interact with a television program in a different way.``\cite{Second-Screen-Def}  Finally, is the second screen poised to become a 
new baseline technology, or does it remain a niche for specialized or vertical applications? 

 \section{Background}

 Second Screen technology has become more known, and used in the past two years, where most of the famous TV prod-casting companies have implemented the technology in either one, or more of their own shows.  Also, as we all know that Xbox, Wii, and PS3 have also have used some compatible devices with second screens in them, to enhance the user experience when interacting with the main screen.
 However, since we briefly defined what a Second Screen is, let us now explain how, and what kind of ways people can use them.  

The first way, Second Screen technology is implemented, "during the live viewing of a show or event, the second screen acts as an additional viewport for the viewer. That means that during an awards show, users can choose to see different camera angles of the venue, access backstage areas and get other content in real-time. Because it's in real-time, this content cannot be repeated for time-shifted viewings (like a DVR)."\cite{Second-Screen-His} Example of this, is as we have seen in 2011 how the Oscar night used the second screen technology, as well as, Grammys, MTV VMAs and the Emmys.\cite{Second-Screen-His}

 The second way of using Second Screen Technology, is "while watching a show or movie, users get a feed of social information ? such as Twitter feeds from the cast or crew, discussions with other viewers and suggestions and electronic programming guides for other types of programming. This data isn't necessarily time-synced with the show itself, but it offers real-time context around what the viewer is doing."\cite{Second-Screen-His} Also, maybe it's important to mention the applications that are using these social discovery features. Some of the most popular apps include Miso http://gomiso.com/ and GetGlue, http://getglue.com/. \cite{Second-Screen-Art} These websites provide the user with applications to download and use in the second device.  Which, allow the user to pick the show he/she want to watch and interact with others who have the same interest, either by messages, videos, voice notes, Tweets, and Facebook comments.

Moreover, we are now seeing that most of the cable providers are now launching their applications for the smartphones, and tablets, for the user to use and watch everything on their second device instead of a standard TV.  For instance, Time Warner Cable, have their amazing application on IOS, and Android Systems, that you can watch TV on your smartphone, or go through the TV guide and search through the available shows, and programs.  Also, they have an interesting tab, which is, DVR, where your phone will work exactly as a remote control and you can record remotely any show you like to watch.  

 Finally, when we all know that a smartphone most of the time is not a convenient method for watching TV and at the same time interacting with the other users.  However, these websites, and cable providers started off their applications on smartphones only, in the years of 2009, and 2010.  However, after testing the satisfaction of their users, they concluded that it's hard to use a second screen application on smartphones, due to their small size screens.  However, by April 2010, Apple released their iPad, and most of the other companies released their tablets.  Therefore, by mid 2011 there came the revolution and the new born of second screen technology, as most of the applications were programmed to work perfectly on those tablets.  Which, resulted in costumer satisfaction and making programming companies direct their focus on making second screen applications.
 % JD: "costumer" ---> "customer"
 \section{Methods} 

For this section, we will be relating to the previously mentioned definitions, and usage of second screen technology, and giving more examples, to better demonstrate the idea of Second Screen Technology. First of all, when we think of second screen applications we think of their capability in making a new  experience for the user.  So, let's see what kind of applications, and technologies out there, and what experience they can provide.  First with the broadcasters, "What are broadcasters doing?" \cite{Second-Screen-Art}  For the broadcasting channels, they're now in a competition with each other, to whom can get the audience a better experience when watching their live shows, such as The voice, America's Got Talent, America's Idol, and the late night shows.
% JD: Make sure to format/capitalize show titles properly (typically italicized, title case). 

Here is at a glance, how NBC envisions their future by " centralizing viewer experience within a centrally branded Second Screen thereby creating greater loyalty to the network and offering viewers a broader social experience than a show specific app.  The app includes more features and an evolving landscape as shows come and go." \cite{Second-Screen-Art}  Here, they are trying to block the available applications in the online stores to spread widely by adding extra features to their own apps.  However, this will open new jobs at the broadcasting companies and they will require bigger budgets to compete with the available applications.

Another Broadcasting application is by NBC called "Fan It (Gamification)." This is a website that NBC has created to make the users engage further with their favorite shows.  Fan it is" s a social networking portal, where fans can log in and make comments about a show; in fact both the app and Fan It provides a single sign-on so that your Second Screen login on NBC Live works on Fan It as well." \cite{Second-Screen-Art}
% JD: OK, some significant typos in the preceding paragraph.
%     Also, so far, your use of direct quotes is too frequent and too long.
%     You should restate things in your own words more frequently.

"NBC uses gamification to drive viewer behavior by dispensing points for participating in events and 
challenges, such as watching videos, sending comments, playing games and performing specific 
activities in a given time. "\cite{Second-Screen-Art}  Moreover with gamification, is that users can earn points, as the more they use the application the more points they get, and then, they can exchange them for NBC goodies.  Finally, with NBC gamification and NBC Live, there's the analytical part that can help them direct their advertisements and "marketing campaigns to target specific interests and 
impulses."\cite{Second-Screen-Art} % JD: Watch your spelling of "gamification" here!

Here is a big thing that everybody is talking about, which is Google TV.  Google has partnered with major companies to make this new era of Second Screen available for the users with higher satisfaction than any other application and devices.  Google TV is either a device connected to the TV, or a TV that has a Google TV built in it.  Google TV provides " thousands of movies, TV episodes, YouTube channels, and apps; with more coming all the time."\cite{Google} Where you can watch, and play anything at any time you like. Also, you can interact with other applications such as, Voice search, Youtube, Google Play, Netflix, and Chrome while you're watching TV, all in one place.  So, this can change the way we know how to TV works. By taking it to a new level where it merges everything we need in one place and making it accessible by either a remote or a phone!
  \section{Discussion}

  As far as I read and thought about Second Screen technology, I find myself agreeing on what most of the broadcasting, cable, and Media companies are saying and expecting from this amazing technology.  However, discussing this topic in terms of possible guidelines, principles, and theories these companies used, is not as clear, and easy as if you were to discuss IOS platform or Android interface system.  However, we see that, a major thing these companies have come to agree on doing, is making everything in one place and accessible to every user the same as the other users have.

 First I want to mention the guideline, "Flexibility for User Control of Data Display" and how Second Screens applications met this particular guideline introduced by (Smith and Mosier 1986).  This guideline focuses on how data are displayed, either in a Screen or any form of data output types.  However, "For effective task performance, displayed data must be relevant to a user's needs"\cite{Guideline} and by that we see that in Second Screen Technology, the user is pretty much have everything relevant to him/her displayed properly on a screen with an easy access and full control, which results in an effective task.  Such as, tweeting while watching America's Got Talent, or while using Google TV, one can access his email and use google music at the same time, and this is of course, makes everything (data) displayed, relevant to the user. 

 The second thing I would like to mention is the "Consistency of Data Display" found in Second Screen technology.  What is consistency of data display?   Consistency of data has to "ensures that each user observes a consistent view of the data, including visible changes made by the user's own transactions and transactions of other users or processes."\cite{guidline2} In this guideline, we can mention how broadcasters of the late shows programs have used this idea of making data consistent.  For example, when you're watching NBC's late show and tweet something, you will see the data you sent (tweet) appear in the application set of user tweets, as well as the late show if the broadcasters pick to show your tweet.  Here the data is consistent.  Also, if you're using Google TV and watch youtube, then write a comment using your phone or computer you will see your comment appear as you watch the video.
    % JD: Speaking of consistency...your capitalization and formatting have been very inconsistent.
    %     You should allocate more time for proofreading.

Finally, from my point of view I see that Second Screen technology is the next big thing.  As, I think it will spread between users, and developers faster than any kind of displayed technologies ever used. As "producers are discovering that a Second Screen experience is quickly becoming mandatory as the audience using mobile devices continue to grow and the aspect of being ?cool? figures into the show?s popularity with a significant section of the audience."\cite{Second-Screen-Art}
    % JD: You have some mis-copied characters in the quote above.

 However, this technology might seem expensive for the companies to implement and monitor, why?  Because for each company to compete with the others, and bring up a good user experience, they have to first build reliable applications, as well as websites or devices to install these applications in.  This will also, require new IT departments to be opened in cable, and broadcast companies, which also, means they need new highly efficient employees to make it work.  "According to Steve Andrade, SVP and GM of Digital Development for NBC.com, the cost for doing one show at a time would be too prohibitive for them. This kind of endeavor requires investment in digital staff for app updates and development, content creation and management, creative integration with the show?s producers and several people to monitor the social conversation and the analytics. This is considered a new division by NBC requiring additional staff." \cite{Second-Screen-Art}

 \section{Conclusion}

In conclusion, as mentioned above Second Screen technology is becoming more used by most of the broadcasters, cable companies, video game producers, and the major leading IT companies.  Why? Because, they greatly enhance the user experience, by bringing everything in one place and making live interaction more enjoyable, and that is what most of the users find convenient. However, at this time, it might be in some cases become a waste of money for some companies to implement this technology, because building and application used by a certain device requires extra money that not all companies can afford easily.  Finally, "the real potential of the second screen is to not just offer up ancillary companion content, but to provide information that is necessary to understand broader story of the show. If Lost were to air today, I'd like to think it would have a heavy second screen component that would include clues and additional information that would be necessary for fully understanding the island."\cite{Second-Screen-His}
 %\subsection{}
 
 \bibliography{Mental-Model-Paper}
\bibliographystyle{alpha}
 
 \end{document}  
