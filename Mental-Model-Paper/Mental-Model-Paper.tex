 \documentclass[12pt, oneside]{article}   	% use "amsart" instead of "article" for AMSLaTeX format
 \usepackage{geometry}                		% See geometry.pdf to learn the layout options. There are lots.
 \geometry{letterpaper}                   		% ... or a4paper or a5paper or ... 
 %\geometry{landscape}                		% Activate for for rotated page geometry
 %\usepackage[parfill]{parskip}    		% Activate to begin paragraphs with an empty line rather than an indent
 \usepackage{graphicx}				% Use pdf, png, jpg, or eps§ with pdflatex; use eps in DVI mode
 								% TeX will automatically convert eps --> pdf in pdflatex	 
	
 \usepackage{amssymb}
 \usepackage{url}
 
 \title{Mental Model Topic Study}
 \author{Abdul Alzaid}
 %\date{}							% Activate to display a given date or no date
 
 \begin{document}
 \maketitle
  \section{Abstract,}
 \section{Introduction}
In this paper we will be discussing, the definition, history, and the latest offerings the new technology that is called Second Screen has come up with. Also, we will discuss how this new technology relate to guidelines and principles of interaction design. Which, with the results of some recent studies, we will conclude if Second Screen technology might, or might not make the current way of interacting with visual media take a new turn to be widely used, and more interactive, and enjoyable by the users in the future. So, what is Second Screen?  "A second screen is a second electronic device used by television viewers to connect to a program they're watching. A second screen is often a smartphone or tablet, where a special complementary app may allow the viewer to interact with a television program in a different way."\cite{Second-Screen-Def}
 \section{Background}
 \section{Methods}
 \section{Discussion}
 \section{Conclusion}


 %\subsection{}
 
 \bibliography{Mental-Model-Paper}
\bibliographystyle{alpha}
 
 \end{document}  
