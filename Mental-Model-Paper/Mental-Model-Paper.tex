 \documentclass[12pt, oneside]{article}   	% use "amsart" instead of "article" for AMSLaTeX format
 \usepackage{geometry}                		% See geometry.pdf to learn the layout options. There are lots.
 \geometry{letterpaper}                   		% ... or a4paper or a5paper or ... 
 %\geometry{landscape}                		% Activate for for rotated page geometry
 %\usepackage[parfill]{parskip}    		% Activate to begin paragraphs with an empty line rather than an indent
 \usepackage{graphicx}				% Use pdf, png, jpg, or eps§ with pdflatex; use eps in DVI mode
 								% TeX will automatically convert eps --> pdf in pdflatex	 
	
 \usepackage{amssymb}
 \usepackage{url}
 
 \title{Mental Model Topic Study}
 \author{Abdul Alzaid}
 %\date{}							% Activate to display a given date or no date
 
 \begin{document}
 \maketitle
  \section{Abstract,}
 \section{Introduction}
In this paper we will be discussing, the definition, history, and the latest offerings the new technology that is called Second Screen has come up with. Also, we will discuss how this new technology relate to guidelines and principles of interaction design. Which, with the results of some recent studies, we will conclude if Second Screen technology might, or might not make the current way of interacting with visual media take a new turn to be widely used, and more interactive, and enjoyable by the users in the future. \paragraph{}
However, what is Second Screen?  "A second screen is a second electronic device used by television viewers to connect to a program they're watching. A second screen is often a smartphone or tablet, where a special complementary app may allow the viewer to interact with a television program in a different way."\cite{Second-Screen-Def}
 \section{Background}
 Second Screen technology has become more known, and used in the past two years, where most of the famous TV prod-casting companies have implemented the technology in either one, or more of their own shows.  Also, as we all know that Xbox, Wii, and PS3 have also have used some compatible devices with second screens in them, to enhance the user experience when interacting with the main screen.
 However, since we briefly defined what a Second Screen is, let us now explain how, and what kind of ways people can use them.  
\paragraph{}
The first way, Second Screen technology is implemented, "during the live viewing of a show or event, the second screen acts as an additional viewport for the viewer. That means that during an awards show, users can choose to see different camera angles of the venue, access backstage areas and get other content in real-time. Because it's in real-time, this content cannot be repeated for time-shifted viewings (like a DVR)."\cite{Second-Screen-His} Example of this, is as we have seen in 2011 how the Oscar night used the second screen technology, as well as, Grammys, MTV VMAs and the Emmys.\cite{Second-Screen-His}
\paragraph{}
 The second way of using Second Screen Technology, is "while watching a show or movie, users get a feed of social information ? such as Twitter feeds from the cast or crew, discussions with other viewers and suggestions and electronic programming guides for other types of programming. This data isn't necessarily time-synced with the show itself, but it offers real-time context around what the viewer is doing."\cite{Second-Screen-His} Also, maybe it's important to mention the applications that are using these social discovery features. Some of the most popular apps include Miso http://gomiso.com/ and GetGlue, http://getglue.com/. \cite{Second-Screen-Art} These websites provide the user with applications to download and use in the second device.  Which, allow the user to pick the show he/she want to watch and interact with others who have the same interest, either by messages, videos, voice notes, Tweets, and Facebook comments.
 \paragraph{}
Moreover, we are now seeing that most of the cable providers are now launching their applications for the smartphones, and tablets, for the user to use and watch everything on their second device instead of a standard TV.  For instance, Time Warner Cable, have their amazing application on IOS, and Android Systems, that you can watch TV on your smartphone, or go through the TV guide and search through the available shows, and programs.  Also, they have an interesting tab, which is, DVR, where your phone will work exactly as a remote control and you can record remotely any show you like to watch.  
 \paragraph{}
 \section{Methods} 
 \section{Discussion}
 \section{Conclusion}


 %\subsection{}
 
 \bibliography{Mental-Model-Paper}
\bibliographystyle{alpha}
 
 \end{document}  
