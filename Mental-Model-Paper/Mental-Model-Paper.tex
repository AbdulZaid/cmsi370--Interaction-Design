 \documentclass[12pt, oneside]{amsart}   	% use "amsart" instead of "article" for AMSLaTeX format
 \usepackage{geometry}                		% See geometry.pdf to learn the layout options. There are lots.
 \geometry{letterpaper}                   		% ... or a4paper or a5paper or ... 
 %\geometry{landscape}                		% Activate for for rotated page geometry
 \usepackage[parfill]{parskip}    		% Activate to begin paragraphs with an empty line rather than an indent
 \usepackage{graphicx}				% Use pdf, png, jpg, or eps§ with pdflatex; use eps in DVI mode
 								% TeX will automatically convert eps --> pdf in pdflatex	 
 \usepackage{amssymb}
 \usepackage{url}
\usepackage{setspace}
\doublespacing 
 \title{Mental Model Topic Study}
 \author{Abdul Alzaid}
 %\date{}							% Activate to display a given date or no date
 \begin{document}
 \maketitle
 \section{Abstract} % JD: LaTeX has an \abstract{} directive specifically for this purpose.
This paper is introducing the idea of the new technology called "Second Screen" and discussing the ideas, and future potentials of this major technology.  Also, this paper is taking in consideration some mental model guidelines while discussing the Second Screen technology.  Moreover, bringing to the readers some quotes by the broadcasters, relating to the future usage and implementation of this new technology into their broadcast world.  As well as, what makes this technology grows rapidly with the major media companies, to the opposite of that which, to what might affect the spread of this technology at this time.

 \section{Introduction}
In this paper we will be discussing, the definition, history, and the latest offerings the new technology that is called Second Screen has come up with. Also, we will discuss how this new technology relate to guidelines and principles of interaction design. Which, with the results of some recent studies, we will conclude if Second Screen technology might, or might not make the current way of interacting with visual media take a new turn to be widely used, and more interactive, and enjoyable by the users in the future. 

% JD: Also, don't forget that LaTeX quotes should be done as `` and ''. (problem fixed)
However, what is Second Screen?  First before answering that question let us bring back its history back into discussion.  Second screen technology is a derivative of  what's called social television. Moreover, a ``social television is a general term for technology that supports communication and social interaction in either the context of watching television, or related to TV content. It also includes the study of television-related social behavior, devices and networks.``\cite{defOfSocialTV}   However, with the study of television-related social behavior, there had to come new services, applications and changes in content, and device productions. Then, came what we call now second screen technology.  Which when defined we say ``a second screen is a second electronic device used by television viewers to connect to a program they're watching. A second screen is often a smartphone or tablet, where a special complementary app may allow the viewer to interact with a television program in a different way.``\cite{Second-Screen-Def}  Finally, is the second screen poised to become a 
new baseline technology, or does it remain a niche for specialized or vertical applications? 

 \section{Background}

 Second Screen technology has become more known, and used in the past two years, where most of the famous TV prod-casting companies have implemented the technology in either one, or more of their own shows.  Also, as we all know that Xbox, Wii, and PS3 have also have used some compatible devices with second screens in them, to enhance the user experience when interacting with the main screen.
 However, since we defined what a Second Screen is, let us now explain how, and what kind of ways people can use them.  Later after what follows we will discuss the question of whether the consumption of parallel or ancillary information really a part of the process of using a second screen? Sometimes, don't people just want to watch something, without navigating around a different set of information?
 
 The first way, Second Screen technology is implemented, "during the live viewing of a show or event, the second screen acts as an additional viewport for the viewer. That means let's say during an awards show, users can choose to see different camera angles of the venue, access backstage areas and get other content in real-time. Because it's in real-time, this content cannot be repeated for time-shifted viewings (like a DVR)."\cite{Second-Screen-His} Example of this, is as we have seen in 2011 how the Oscar night used the second screen technology, as well as, Grammys, MTV VMAs and the Emmys.\cite{Second-Screen-His} But, is this really what people want when they watch an important event? Maybe they just don't want any interruption of data consistency while enjoying such an event.

However, the second way of using Second Screen technology, is "while watching a show or movie, users get a feed of social information. Such as, Twitter feeds from the cast or crew, discussions with other viewers and suggestions and electronic programming guides for other types of programming. This data isn't necessarily time-synced with the show itself, but it offers real-time context around what the viewer is doing."\cite{Second-Screen-His} Also, maybe it's important to mention the applications that are using these social discovery features. Some of the most popular apps include Miso http://gomiso.com/ and GetGlue, http://getglue.com/. \cite{Second-Screen-Art} These websites provide the user with applications to download and use in the second device.  Which, allow the user to pick the show he/she wants to watch and interact with others who have the same interest, either by messages, videos, voice notes, Tweets, and Facebook comments.

Moreover, we are now seeing that most of the cable providers are now launching their applications for the smartphones, and tablets, for the user to use and watch everything on their second device instead of a standard TV.  For instance, Time Warner Cable, have their amazing application on IOS, and Android Systems, that you can watch TV on your smartphone, or go through the TV guide and search through the available shows, and programs.  Also, they have an interesting tab, which is, DVR, where your phone will work exactly as a remote control and you can record remotely any show you like to watch.  

 Finally, when we all know that a smartphone most of the time is not a convenient method for watching TV and at the same time interacting with the other users.  However, these websites, and cable providers started off their applications on smartphones only, in the years of 2009, and 2010.  However, after testing the satisfaction of their users, they concluded that it's hard to use a second screen application on smartphones, due to their small size screens.  However, by April 2010, Apple released their iPad, and most of the other companies released their tablets.  Therefore, by mid 2011 there came the revolution and the new born of second screen technology, as most of the applications were programmed to work perfectly on those tablets.  Which, resulted in customer satisfaction and making programming companies direct their focus on making second screen applications.  On the other hand, the question of whether it enhances the experience or actually takes away from being fully invested in the storyline when using second screen technology is still will be discussed in the discussion part of this paper.
 \section{Visuals} 
Here in this section is a set of images that demonstrate how Second Screen technology look like.
\begin{figure}[ht!]
\centering
\includegraphics[width=85mm]{/Users/AbdulZaid/Downloads/Chris_Bennitt.jpg}
\caption{A simple caption for demonstrating how Second Screen looks like By: Chris Bennit}
\label{overflow}
\end{figure}

Here's another image that show some usage of Second Screen,
\begin{figure}[ht!]
\centering
\includegraphics[width=85mm]{/Users/AbdulZaid/Downloads/BrightCove.jpg}
\caption{A simple caption for demonstrating how Second Screen looks like By: www.brightcove.com}
\label{overflow}
\end{figure}

Here is a last demonstration of Second Screen,
\begin{figure}[ht!]
\centering
\includegraphics[width=85mm]{/Users/AbdulZaid/Downloads/Ankur.jpg}
\caption{A simple caption for demonstrating how Second Screen looks like By: Ankur Angal}
\label{overflow}
\end{figure}

\section{Methods} 
For this section, we will be relating to the previously mentioned definitions, and usage of second screen technology, and giving more examples, to better demonstrate the idea of Second Screen technology. First of all, when we think of second screen applications we think of their capability in making a new experience for the user.  So, let's see what kind of applications, and technologies out there, and what experience they can provide.  First with the broadcasters, "what are broadcasters doing?" \cite{Second-Screen-Art}  For the broadcasting channels, they're now in a competition with each other, to whom can get the audience a better experience when watching their live shows, such as \emph{The Voice, America's Got Talent, America's Idol}, and the \emph{Late Night shows.}

Here is at a glance, how NBC envisions their future by " centralizing viewer experience within a centrally branded Second Screen, thereby creating greater loyalty to the network and offering viewers a broader social experience than a show specific app.  The app includes more features and an evolving landscape as shows come and go." \cite{Second-Screen-Art}  Here, they are trying to block the available applications in the online stores to spread widely by adding extra features to their own apps.  However, this will open new jobs at the broadcasting companies and they will require bigger budgets to compete with the available applications.

Another Broadcasting application is by NBC called "Fan It (Gamification)." This is a website that NBC has created to make the users engage further with their favorite shows.  Fan it is "a social networking portal, where fans can log in and make comments about a show; in fact both the app and Fan It provides a single sign-on so that your Second Screen login on NBC Live works on Fan It as well." \cite{Second-Screen-Art}

"NBC uses gamification to drive viewer behavior by dispensing points for participating in events and 
challenges, such as watching videos, sending comments, playing games and performing specific 
activities in a given time. "\cite{Second-Screen-Art}  Moreover with gamification, is that users can earn points, as the more they use the application the more points they get, and then, they can exchange them for NBC goodies.  Finally, with NBC gamification and NBC Live, there's the analytical part that can help them direct their advertisements and marketing campaigns to target specific interests and 
users' feedback.

Here is a big thing that everybody is talking about, which is  \emph{Google TV}.  Google has partnered with major companies to make this new era of Second Screen available for the users and provide higher satisfaction than any other application and devices.   \emph{Google TV} is either a device connected to the TV, or a TV that has a  \emph{Google TV} built in it.  Google TV provides " thousands of movies, TV episodes, YouTube channels, and apps; with more coming all the time."\cite{Google} Where you can watch, and play anything at any time you like.  Also, you can interact with other applications such as,  \emph{Voice search, Youtube, Google Play, Netflix"},  and  \emph{Chrome} while you're watching TV, all in one place.  So, this can change the way we know how TV works. By taking it to a new level where it merges everything we need in one place and making it accessible by either a remote or a phone.

 \section{Discussion}
 In this part of discussion we will discuss the most important questions that should be answered regarding Second Screen in terms of interface design metrics, guidelines, and principles.  However, some of the most important questions ask whether everyone can use a Second Screen interface? Can one make mistakes easily? Does the second screen detract from the primary viewing experience?  Finally, is the second screen really how everyone would like to view their video entertainment? Are people who have "just" a single screen now at a disadvantage? Is it appropriate to all types of entertainment?  Furthermore in this section I will mention my point of view to all of this and what I think of Second Screen, and what its future can be?
 
 The first questions of whether everyone can use a Second Screen interface or not?  Well, to answer this question, let's first realize that some people are entitled to their current methods of using the interfaces available.  For example, if you are a person who's age is over 50 and you have been introduced to this technology for the first time.  Of course, your reaction to it will be negative, as well as any disabled people who are incapable of interacting with such interfaces that required more than one device to run.  Moreover, younger children will also be incapable of running those interfaces, because they require more knowledge than just clicking a remote to change a channel. 
 
 For the second question, can one make mistakes easily?  This concerns with the metric  \emph{error making}. However, the answer to this question is definitely yes, as any other new interface should be.  Here are some of the errors the users always make.  The first error is connecting the second device to the first one, it can be confusing at first when users are not familiar with how to use the interface application or how to get the content to show on the other screen.  Also, from my own experience when I use the \emph{Youtube} application on my  \emph{iPhone} and connect it to my  \emph{Playstation} to play  some videos on the TV, I always go and engage with my phone while everyone is watching the video that's playing,  and suddenly without noticing I tend to open the \emph{Youtube} application and hit the back or search button to search for something else, and it turns out that it changes what everybody is watching on the first screen.  So, yes Second Screen technology is not an error free, but quite the opposite.
 
 Moreover, does the second screen detract from the primary viewing experience?  the answer to this question is certainly yes, for the following reasons.  Let us recall what we said about the 2011 \emph{Oscar} awards that used Second Screen, and how it allow the user to change camera angels and read comments and tweets while watching the primary show.  This at first seems interesting, but when you actually think of it, it detract the user from the primary show especially when this technology is being used  by the user for the first time.  The user will tend to experience the abilities of this "cool" technology and detract from the primary show for the sake of exploring it.  However, you must note that this kind of features are only available to explore at the time of the show.  So, not all the available shows will allow you to use these features, therefore, there is no time for you to explore these camera angel features before watching your primary show. 
 
 Furthermore, when you watch a Live show, such as, talent shows, signing shows, or late night shows, and during the time of the show you decided to go and explore Live tweets, comments, or Ads the Second Screen device possess.  This will take your attention away from the prime Live show and result in making you miss so many important events you were excited to see, but you couldn't because you got attached to an Ad or a tweet\cite{BenifitsAndImpacts}.  Therefore, yes Second Screen apply the guideline of "consistency of data display."  But, that in this case is not what the user needs which results to the answer of the question, which is definitely Yes. 
  
  An important question needs to be answered which asks, is the second screen really how everyone would like to view their video entertainment? Are people who have "just" a single screen now at a disadvantage?  Let us answer the second questions first as it leads to answering the first one.  Being with a single screen at our time now is for sure not a disadvantage, but it will be in the next 5 years or so.  However, being with a single screen is a good thing for a lot of people, specially those who are not in a favor of multitasking while engaging in a video entertainment.  I am one of them, as I can't really watch a show and engage in anything else during that time, not even engaging with a human conversation. Therefore, I always prefer a single screen to show the content I need to watch rather than lots of content into two screens.  Imagine going to the movies and they use Second Screen technologies there! Certainly not going to be entertaining. So, people with single screen are not in disadvantage, as Second Screen doesn't add that much to the content but rather sometimes takes it away by taking the user attention away, and that will also answer the first question by No.
  
 Moreover, let us summarize Second Screen in terms of metrics.  Yes we answered so many questions that point out the answers to most of the metrics introduced by "Nielsen".  Are they effective? Mostly Yes, they are  effective in accomplishing their task which is allowing the user to engage in a wide range of social interaction while watching TV, playing video game, or any sort of entertainment.  Moreover, the speed of performance is relatively fast when using Second Screen, as you can tweet, watch an Ad, or vote at the same time you're using the first screen.
 
 More on this, the time to learn (learnability) is in what most studies show it is not that hard to learn how to use a  Second Screen, and that the incredibly increased number of people using it day by day proves that it is easy to learn, and that "a second screen is not difficult in most situations"\cite{AdvDisv}.  Furthermore, when it comes to (satisfaction), a study was conducted on groups of people that showed the results of accomplishing certain tasks by using Second Screens, were positive and that users accomplished their tasks "more quickly and more 
efficiently than users with single screens."\cite{Dell} Therefore, Second Screens " increased productivity and user satisfaction."\cite{Dell} 
 
 On the other hand, as far as I read and thought about Second Screen technology, I find myself agreeing on what most of the broadcasting, cable, and Media companies are saying and expecting from this amazing technology.  However, discussing this topic in terms of possible guidelines, principles, and theories these companies used, is not as clear, and easy as if you were to discuss IOS platform or Android interface system.  However, we see that, a major thing these companies have come to agree on doing, is making everything in one place and accessible to every user the same as the other users have.

 First I want to mention the guideline, "Flexibility for User Control of Data Display" and how Second Screens applications met this particular guideline introduced by (Smith and Mosier 1986).  This guideline focuses on how data are displayed, either in a Screen or any form of data output types.  However, "For effective task performance, displayed data must be relevant to a user's needs"\cite{Guideline} and by that we see that in Second Screen Technology, the user is pretty much have everything relevant to him/her displayed properly on a screen with an easy access and full control, which results in an effective task.  Such as, tweeting while watching America's Got Talent, or while using Google TV, one can access his email and use google music at the same time, and this is of course, makes everything (data) displayed, relevant to the user. 

 The second thing I would like to mention is the "Consistency of Data Display" found in Second Screen technology.  What is consistency of data display?   Consistency of data has to "ensures that each user observes a consistent view of the data, including visible changes made by the user's own transactions and transactions of other users or processes."\cite{guidline2} In this guideline, we can mention how broadcasters of the late shows programs have used this idea of making data consistent.  For example, when you're watching NBC's late show and tweet something, you will see the data you sent (tweet) appear in the application set of user tweets, as well as the late show if the broadcasters pick to show your tweet.  Here the data is consistent.  Also, if you're using  \emph{Google TV} and watch  \emph{Youtube}, then write a comment using your phone or computer you will see your comment appear as you watch the video.


Finally, from my point of view I see that Second Screen technology is the next big thing.  As, I think it will spread between users, and developers faster than any kind of displayed technologies ever used. As "producers are discovering that a Second Screen experience is quickly becoming mandatory as the audience using mobile devices continue to grow and the aspect of being "cool" figures into the show's popularity with a significant section of the audience."\cite{Second-Screen-Art}

 However, this technology might seem expensive for the companies to implement and monitor, why?  Because for each company to compete with the others, and bring up a good user experience, they have to first build reliable applications, as well as websites or devices to install these applications in.  This will also, require new IT departments to be opened in cable, and broadcast companies, which also, means they need new highly efficient employees to make it work.  "According to Steve Andrade, SVP and GM of Digital Development for NBC.com, the cost for doing one show at a time would be too prohibitive for them. This kind of endeavor requires investment in digital staff for app updates and development, content creation and management, creative integration with the show?s producers and several people to monitor the social conversation and the analytics. This is considered a new division by NBC requiring additional staff." \cite{Second-Screen-Art}

 \section{Conclusion}

In conclusion, as mentioned above Second Screen technology is becoming more used by most of the broadcasters, cable companies, video game producers, and the major leading IT companies.  Why? Because, they greatly enhance the user experience, by bringing everything in one place and making live interaction more enjoyable, and that is what most of the users find convenient, but not all. Because, we found out that Second Screens are not suitable for everyone, and it is not error free interfaces.  Also, we found out that they can become a distraction rather than enhancement to some of the users.  However, we also found out that they are becoming highly satisfiable and more implemented as times goes.  Moreover, at this time, it might be in some cases become a waste of money for some companies to implement this technology, because building an application used by a certain device requires an extra money that not all companies can afford easily. Finally, think of "the real potential of the second screen [as] is to not just offer up ancillary companion content, but to provide information that is necessary to understand broader story of the show. If Lost were to air today, I'd like to think it would have a heavy second screen component that would include clues and additional information that would be necessary for fully understanding the island."\cite{Second-Screen-His}
 %\subsection{}
 
 \bibliography{Mental-Model-Paper}
\bibliographystyle{alpha}
 
 \end{document}  
