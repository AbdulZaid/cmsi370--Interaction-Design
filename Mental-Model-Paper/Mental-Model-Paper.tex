 \documentclass[12pt, oneside]{article}   	% use "amsart" instead of "article" for AMSLaTeX format
 \usepackage{geometry}                		% See geometry.pdf to learn the layout options. There are lots.
 \geometry{letterpaper}                   		% ... or a4paper or a5paper or ... 
 %\geometry{landscape}                		% Activate for for rotated page geometry
 %\usepackage[parfill]{parskip}    		% Activate to begin paragraphs with an empty line rather than an indent
 \usepackage{graphicx}				% Use pdf, png, jpg, or eps§ with pdflatex; use eps in DVI mode
 								% TeX will automatically convert eps --> pdf in pdflatex	 
	
 \usepackage{amssymb}
 \usepackage{url}
\usepackage{setspace}
\doublespacing 
 \title{Mental Model Topic Study}
 \author{Abdul Alzaid}
 %\date{}							% Activate to display a given date or no date
 
 \begin{document}
 \maketitle
  \section{Abstract,}
 \section{Introduction}
\paragraph{}
In this paper we will be discussing, the definition, history, and the latest offerings the new technology that is called Second Screen has come up with. Also, we will discuss how this new technology relate to guidelines and principles of interaction design. Which, with the results of some recent studies, we will conclude if Second Screen technology might, or might not make the current way of interacting with visual media take a new turn to be widely used, and more interactive, and enjoyable by the users in the future. \paragraph{}
However, what is Second Screen?  "A second screen is a second electronic device used by television viewers to connect to a program they're watching. A second screen is often a smartphone or tablet, where a special complementary app may allow the viewer to interact with a television program in a different way."\cite{Second-Screen-Def}
 \section{Background}
 \paragraph{}
 Second Screen technology has become more known, and used in the past two years, where most of the famous TV prod-casting companies have implemented the technology in either one, or more of their own shows.  Also, as we all know that Xbox, Wii, and PS3 have also have used some compatible devices with second screens in them, to enhance the user experience when interacting with the main screen.
 However, since we briefly defined what a Second Screen is, let us now explain how, and what kind of ways people can use them.  
\paragraph{}
The first way, Second Screen technology is implemented, "during the live viewing of a show or event, the second screen acts as an additional viewport for the viewer. That means that during an awards show, users can choose to see different camera angles of the venue, access backstage areas and get other content in real-time. Because it's in real-time, this content cannot be repeated for time-shifted viewings (like a DVR)."\cite{Second-Screen-His} Example of this, is as we have seen in 2011 how the Oscar night used the second screen technology, as well as, Grammys, MTV VMAs and the Emmys.\cite{Second-Screen-His}
\paragraph{}
 The second way of using Second Screen Technology, is "while watching a show or movie, users get a feed of social information ? such as Twitter feeds from the cast or crew, discussions with other viewers and suggestions and electronic programming guides for other types of programming. This data isn't necessarily time-synced with the show itself, but it offers real-time context around what the viewer is doing."\cite{Second-Screen-His} Also, maybe it's important to mention the applications that are using these social discovery features. Some of the most popular apps include Miso http://gomiso.com/ and GetGlue, http://getglue.com/. \cite{Second-Screen-Art} These websites provide the user with applications to download and use in the second device.  Which, allow the user to pick the show he/she want to watch and interact with others who have the same interest, either by messages, videos, voice notes, Tweets, and Facebook comments.
 \paragraph{}
Moreover, we are now seeing that most of the cable providers are now launching their applications for the smartphones, and tablets, for the user to use and watch everything on their second device instead of a standard TV.  For instance, Time Warner Cable, have their amazing application on IOS, and Android Systems, that you can watch TV on your smartphone, or go through the TV guide and search through the available shows, and programs.  Also, they have an interesting tab, which is, DVR, where your phone will work exactly as a remote control and you can record remotely any show you like to watch.  
 \paragraph{}
 Finally, when we all know that a smartphone most of the time is not a convenient method for watching TV and at the same time interacting with the other users.  However, these websites, and cable providers started off their applications on smartphones only, in the years of 2009, and 2010.  However, after testing the satisfaction of their users, they concluded that it's hard to use a second screen application on smartphones, due to their small size screens.  However, by April 2010, Apple released their iPad, and most of the other companies released their tablets.  Therefore, by mid 2011 there came the revolution and the new born of second screen technology, as most of the applications were programmed to work perfectly on those tablets.  Which, resulted in costumer satisfaction and making programming companies direct their focus on making second screen applications.
 \section{Methods} 
 \paragraph{}
For this section, we will be relating to the previously mentioned definitions, and usage of second screen technology, and giving more examples, to better demonstrate the idea of Second Screen Technology. First of all, when we think of second screen applications we think of their capability in making a new  experience for the user.  So, let's see what kind of applications, and technologies out there, and what experience they can provide.  First with the broadcasters, "What are broadcasters doing?" \cite{Second-Screen-Art}  For the broadcasting channels, they're now in a competition with each other, to whom can get the audience a better experience when watching their live shows, such as The voice, America's Got Talent, America's Idol, and the late night shows.  
 \paragraph{}
Here is at a glance, how NBC envisions their future by " centralizing viewer experience within a centrally branded Second Screen thereby creating greater loyalty to the network and offering viewers a broader social experience than a show specific app.  The app includes more features and an evolving landscape as shows come and go." \cite{Second-Screen-Art}  Here, they are trying to block the available applications in the online stores to spread widely by adding extra features to their own apps.  However, this will open new jobs at the broadcasting companies and they will require bigger budgets to compete with the available applications.
 \paragraph{}
Another Broadcasting application is by NBC called "Fan It (Gamification)." This is a website that NBC has created to make the users engage further with their favorite shows.  Fan it is" s a social networking portal, where fans can log in and make comments about a show; in fact both the app and Fan It provides a single sign-on so that your Second Screen login on NBC Live works on Fan It as well." \cite{Second-Screen-Art}
  \paragraph{}
"NBC uses gamification to drive viewer behavior by dispensing points for participating in events and 
challenges, such as watching videos, sending comments, playing games and performing specific 
activities in a given time. "\cite{Second-Screen-Art}  Moreover with gamfication, is that users can earn points, as the more they use the application the more points they get, and then, they can exchange them for NBC goodies.  Finally, with NBC gamficication and NBC Live, there's the analytical part that can help them direct their advertisements and "marketing campaigns to target specific interests and 
impulses."\cite{Second-Screen-Art}
  \paragraph{}
Here is a big thing that everybody is talking about, which is Google TV.  Google has partnered with major companies to make this new era of Second Screen available for the users with higher satisfaction than any other application and devices.  Google TV is either a device connected to the TV, or a TV that has a Google TV built in it.  Google TV provides " thousands of movies, TV episodes, YouTube channels, and apps; with more coming all the time."\cite{Google} Where you can watch, and play anything at any time you like. Also, you can interact with other applications such as, Voice search, Youtube, Google Play, Netflix, and Chrome while you're watching TV, all in one place.  So, this can change the way we know how to TV works. By taking it to a new level where it merges everything we need in one place and making it accessible by either a remote or a phone!
  \section{Discussion}
 \section{Conclusion}


 %\subsection{}
 
 \bibliography{Mental-Model-Paper}
\bibliographystyle{alpha}
 
 \end{document}  
