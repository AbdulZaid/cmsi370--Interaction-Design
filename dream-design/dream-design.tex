\documentclass[12pt, oneside]{amsart}   	% use "amsart" instead of "article" for AMSLaTeX format
\usepackage{geometry}                		% See geometry.pdf to learn the layout options. There are lots.
\geometry{letterpaper}                   		% ... or a4paper or a5paper or ... 
\usepackage{setspace}
%\geometry{landscape}                		% Activate for for rotated page geometry
%\usepackage[parfill]{parskip}    		% Activate to begin paragraphs with an empty line rather than an indent
\usepackage{graphicx}				% Use pdf, png, jpg, or eps§ with pdflatex; use eps in DVI mode
								% TeX will automatically convert eps --> pdf in pdflatex	
	
\usepackage{amssymb}

\title{Dream Design}
\author{Abdul Alzaid}
%\date{}							% Activate to display a given date or no date

\begin{document}
\maketitle
%\section{}
%\subsection{}
\section{Introduction}
\doublespacing

Interfaces and their designs are considered a necessity in our daily technological use of modules that can be applied on a hardware, or a software.  Moreover, the interface between a user and a machine, has to be built based on some guidelines, principles, and theories of interface designs, that allow the user to have a better experience when using those interfaces.  However, in this paper I will mention my dream interface that can be applied as a software in a specifically designed machine, which therefore has to follow certain guidelines and principles of design that are going to be mentioned. So what is my dream interface?

Since we are all surrounded by so many hard objects in our lives, and those objects sometimes are needed to be shown exactly as they are on a computer or a tablet so they can be illustrated or modified in a way or another. Therefore, what I have always dreamed of is a 3D scanner from your own computer, tablet, or phone's screen. The screen can work as a scanner which scans any object you place it in front of it, to an exact 3D model of it in the software that is being used. Moreover, the screen will capture objects in full color with multi-laser precision. The laser comes out from the screen and scans the object and deliver a 3D model of that object to your computer. However, that is jot say we are not focusing on the 3D scanner itself and its functionality, but rather focusing on a dream well designed user interface for that 3D scanner. Moreover, we should know that the purpose of the user interface is to make the user's interaction as simple and efficient as possible, in terms of accomplishing the user's goals.  Moreover, the user interface design should account for its users' mental model, and also be more adaptable to changing user needs while performing the task.  Finally, the dream interface design should also be based on interaction design principles, to deliver a preferable user interface for most of users.
% JD: A word of caution at this point: remember that this assignment is about a dream
%     *design* either on a well-known application or a real-world simulation.  I don't
%     think a 3D scanner matches either category.  But I'll ride with it as long as
%     you do design a user interface.
\section{Design}

The 3D scanner interface design should be implemented in every device that has a screen in it.  For example, PCs, tablets, Televisions, smart-phones, and video game devices. Furthermore, the interface design should be efficient, and easy to use by all kind of users. Therefore, it should make the user interaction with the machine easier and more convenient than ever in accomplishing this kind of tasks. 

Furthermore, we all know that there are some 3D printers out there, and few 3D scanners as well.  However, the 3D scanners that are available now are very expensive and they are in the form of a different device that is connected to a computer and require a whole workstation to perform their task.  Also, the ones available now needs more than one positioning to the object as shown in the next image

\begin{figure}[ht!]
\centering
\includegraphics[width=90mm]{../../../Downloads/photo41.jpg}
\caption{A simple caption by: makerbot.com}
\label{overflow}
\end{figure}

However, let us talk first about the 3D scanner and how it can function, to better understand the idea before we dive into the dream user interface design itself, which this 3D scanner uses to perform its tasks.
First, the 3D scanner I am thinking of is actually within the screen itself.  For example, think of your laptop screen as it's able to fire some laser when it asked to do so by a command that is in the software drop down menu, or a shortcut similar to "command + P" for printing. Then, the object placed in front of it, let us say on the user's hand gets scanned in the form of a 3D model to your computer.  Also, when 3D scanner is used within the most used device by the user, it makes the user more comfortable and only needs simple instructions on how to use this kind of scanner, which is the point where it clearly leads to my dream design interface that needs to be designed in any computer, tablet, or smartphone, to allow the user to have a better interaction with the 3D scanner.
% JD: OK, the danger zone is increasing...it's beginning to sound like you have not
%     come up with a dream *design* for a 3D scanner application, but *you are talking
%     about the scanner functionality itself*.  This is the classic "utility vs. usability"
%     pitfall.  We want a *new user interface* on a well-known or simulated application
%     category, not a new *application* in itself.


My dream interface design is a software that can be installed in any computer, tablet, or a smart-phone that uses a modern system.  The software I am dreaming of in accomplishing such tasks, starts off by introducing the user on how the software runs and what each button can do.  That is done by either showing a step by step introductory, or by showing a video the user can go through to see how this interface works.  

Moreover, the interface should feel and look like a real scanner.  How?  Well, let us first imagine how installing a new software into a system looks like. The installing process is shown exactly on the screen while the system is working on installing it, and that is done by showing a Live window that contains information that tells the user at what step the system is, and how many minutes are left for the installing process to be done.   However, my dream interface should do the same, but by showing the user exactly what is being scanned and what is not during the process of scanning and that will allow the user to accomplish the task with a responsive system showing what needs to be done.  

More, 
\section{Usage Scenarios}

When it comes to the usage scenarios of this interface, there are many.  For instance, making movies requires implementing real objects into them. Let us say you are a 3D Modeler that needs to design some kind of buildings, cars, weapons, and you only have the smaller version of that object on your hand. Usually visual artist will place that object and draw it using some kind of application in their computers. Of course, this takes time with less precision.  But, with this kind of interface the only thing you need to do is hold that object in front of the computer's screen and it will laser scan the object with hundred percent precision.  

Another usage, we all have objects in hand, and sometimes we complain about their design, layout, and maybe functionality. For example, you're holding a glasses or sunglasses, and that glasses doesn't fit very well, or maybe it can break easily and you have some adjustment in mind. So, what you do is scan it and then edit it.  When you're done, send a copy of your work to the manufacturer and they will get you what you asked for.  Moreover, everyone has things in mind and don't know how or where to suggest their ideas.  So, this kind of interface if used with a specified application that allow the user to edit the 3D model easily, and link the application to any company that manufactures the product, then I believe this will revolutionize the industry with so many innovations and adjustments to the current objects we have.  
% JD: Yes, it's really looking like it.  You are talking about a dream *application*,
%     not a new user interface for a pre-existing or well-known one.

\section{Rationale}

For this section we will discuss the interaction design concepts that are used in this dream interface. The first thing about a good interface is that it should facilitate finishing the task at hand without making unnecessary attention to itself, which this dream interface does. Also, a user interface design requires a good understanding of user needs, and we all need this 3D scanner to be available at any time to manage our tasks easily, as demonstrated in the previous section. 
% JD: What concept, precisely, does "facilitate finishing the task at hand" represent?
%     I don't recall this.  User needs is established, yes, but this can also refer to
%     requirements gathering.  You should use established terms and phrases for your
%     rational, so that they are automatically, recognizably taken from the interaction
%     design field.

Moreover, this dream interface should be effective, efficient, and satisfactory.
% JD: Effectiveness is not a metric.
The interface should be effective because it scans the object into a 3D model of it without using any other devices to do this task. This interface allow the user to achieve its goals of making a 3D model of anything in hand with multiple clicks, instead of a whole set of devices and stations.  Also, it can be very efficient as it requires nothing but for the user to hold the object in front of the screen, and that will definitely reflect on the overall user satisfaction to accomplish a task that is considered really hard in our days.  Furthermore, these quality factors of usability can be affected to be better by the simplicity of this design if manufactured as intended to be, which in a way that the design will not use unnecessary complexity. 
% JD: There is hardly any interface to talk about here...you are focused on *features*
%     (what the user can do) and not *interface* (how the user communicates with the
%     machine in order to do something).

Finally, the interface should measure for the predictability factor in it.  For example, when the laser comes out the user needs to know that he/she needs to rotate the object once the laser stay still, which means it's time for the user to rotate the object to a different angle to allow the scanner to capture the whole object into 3D model.
Moreover, the interface should also be responsive and provides enough feedback information about the system status and the task completion, while the user is making the task.
% JD: OK, this last paragraph is more in the desired direction, but still, you are
%     describing what the interface for a 3D scanner *should be*, but *not the interface
%     itself*.  You say it should be responsive---how?  What does it do?  What response
%     does it show?  You ask for feedback---what exactly?  How does that feedback look?
%     When does it appear?  *That* is the design, and aside from saying that you put an
%     object in front of the computer to scan it, there is nothing else (so far) that
%     describes how this is done.
\section{Usability Metrics}

For this section we will assume that the dream interface is being implemented and tested, then what are its strongest Usability Metrics?  Well, I believe once this interface is being tested the stodgiest metric will be efficiency as mentioned above.  The reason why, is that this kind of task requires a lot of time to accomplish and also more than one device. But, in this interface, it should only take up to two minutes to accomplish 2 hours task.  The next strong metric should be learnability, but only under one condition which as we mentioned before the system should be responsive.  Therefore, the user will only need to move the object as directed, and that requires no further, or previous learning. 
% JD: The efficiency discussion mixes up efficiency of functionality (scanning speed
%     has nothing to do with the user interface) and efficiency of user interface.
%     We want the latter.
%
%     Learnability is also misapplied: remember that learnability is about how quickly
%     the user figures things out.  You seem to be saying that because moving an object
%     is the only thing that the user does in this application, then only one thing must
%     be learned and therefore the interface (such that it is) is learnable.  But, see---
%     that is *not* what learnability refers to.  With a bad design, the user might take
%     a long time to figure out that the 3D-scanned object needs to be moved...and
%     unfortunately you don't *have* a design, good or bad, against which to make this
%     determination or forecast.

However, this system can be weak in some of the metrics, and I am thinking it should be error making.  The reason is that this kind of interface needs to capture precise data, and that requires precise movement, and placement to the object in front of the screen.  Therefore, users may place the object too close or too far from the scanning area "let's say above the keyboard or 3 inches away from the screen if it were a tablet."  Also, some will definitely if not all, not going to rotate the object of not instructed to so. Whereas, some will rotate the object but not in all angles.  
% JD: So here, yes, we see some genuine potential usability errors---the user not knowing
%     where to place an object; the user not knowing how to rotate it.  That's fine, but...
%     where is your interface?  There is no way to tell how this system might perform with
%     errors because there is no description of the user interface of the system!  How does
%     the application prevent the user from positioning an object too close?  How does the
%     application make sure that the object is turned when needed?  Those are the issues
%     to cover.

Finally, this kind of interface should be very satisfactory to the users as it's a new and a very important innovation, that will require a lot of research and hardware new technologies in order to get it to work as it's planned to be.
% JD: Sure, the *application* is innovative.  But that is the application, not the interface.

\section{Target Systems}

Before, we sum up all this dream interface, we should mention what kind of system this interface can work on.  I believe this interface needs to be implemented in any photo, or movie editor as a choice from a drawdown menu or any sort of functional button.  Also, there should be a system such as photoshop, or let's call it 3D shop, that you need to run and simultaneously work with you as you scan the object.  Also, this 3D shop should allow the user to adjust the scanned object, either by changing its color, shape, and everything. Moreover, like we mentioned above, this system should be linked online to the manufacturing companies or any sort of online community that focuses on editing 3D objects, and it should allow the user to send feedback and edits to those companies.  Also, this system should be downloadable in any computer, phone, tablet or gaming device, such that it allow the user to use his/her models anywhere they go. 
% JD: This paragraph continues to mix up "utility" and "usability."  Here is another way to
%     state it: you should focus on *how the user communicates with the machine* to perform
%     a particular task, *and not the task itself*.
\section{Conclusion}

Finally, I believe this dream interface, can become a reality in the near future as there are so many research companies that tries to advance the current 3D scanners into smaller and more compatible ones.  Furthermore, the need for this kind of interface will accelerate the wheel towards making it possible in the near future. Finally, due to its importance and efficiency in terms of its usage, this interface will be the next big thing in education, art, movie making, video games playing, and manufacturing in general.
% JD: Note again that this is focused on what the system lets the user do, and now how the
%     user tells the system to do it.
\end{document}  

% JD: No citations?  For technology that is this speculative, I would expect that you have
%     at least one reference.  Otherwise, this is just science fiction.
