\documentclass[12pt, oneside]{amsart}   	% use "amsart" instead of "article" for AMSLaTeX format
\usepackage{geometry}                		% See geometry.pdf to learn the layout options. There are lots.
\geometry{letterpaper}                   		% ... or a4paper or a5paper or ... 
\usepackage{setspace}
%\geometry{landscape}                		% Activate for for rotated page geometry
%\usepackage[parfill]{parskip}    		% Activate to begin paragraphs with an empty line rather than an indent
\usepackage{graphicx}				% Use pdf, png, jpg, or eps§ with pdflatex; use eps in DVI mode
								% TeX will automatically convert eps --> pdf in pdflatex	
\usepackage{float}
\floatstyle{boxed}
\restylefloat{figure}	
\usepackage{amssymb}

\title{Dream Design}
\author{Abdul Alzaid}
%\date{}							% Activate to display a given date or no date

\begin{document}
\maketitle
%\section{}
%\subsection{}
\section{Introduction}
\doublespacing

Interfaces and their designs are considered a necessity in our daily technological use of modules that can be applied on a hardware, or a software.  Moreover, the interface between a user and a machine, has to be built based on some guidelines, principles, and theories of interface designs, that allow the user to have a better experience when using those interfaces.  However, in this paper I will mention my dream interface that can be applied as a software in a specifically designed machine, which therefore has to follow certain guidelines and principles of design that are going to be mentioned. So what is my dream interface?

Since we are all surrounded by so many hard objects in our lives, and those objects sometimes are needed to be shown exactly as they are on a computer or a tablet so they can be illustrated or modified in a way or another. Therefore, what I have always dreamed of is a 3D scanner from your own computer, tablet, or phone's screen. The screen can work as a scanner which scans any object you place it in front of it, to an exact 3D model of it in the software that is being used. Moreover, the screen will capture objects in full color with multi-laser precision. The laser comes out from the screen and scans the object and deliver a 3D model of that object to your computer. However, that is jot say we are not focusing on the 3D scanner itself and its functionality, but rather focusing on a dream well designed user interface for that 3D scanner. Moreover, we should know that the purpose of the user interface is to make the user's interaction as simple and efficient as possible, in terms of accomplishing the user's goals.  Moreover, the user interface design should account for its users' mental model, and also be more adaptable to changing user needs while performing the task.  Finally, the dream interface design should also be based on interaction design principles, to deliver a preferable user interface for most of users.

\section{Design}
The 3D scanner interface design should be implemented in every device that has a screen in it.  For example, PCs, tablets, Televisions, smart-phones, and video game devices. Furthermore, the interface design should be efficient, and easy to use by all kind of users. Therefore, it should make the user interaction with the machine easier and more convenient than ever in accomplishing this kind of tasks. 

Furthermore, we all know that there are some 3D printers out there, and few 3D scanners as well.  However, the 3D scanners that are available now are very expensive and they are in the form of a different device that is connected to a computer and require a whole workstation to perform their task.  Also, the ones available now needs more than one positioning to the object as shown in the next image

\begin{figure}[ht!]
\centering
\includegraphics[width=150mm]{../../../Downloads/photo41.jpg}
\caption{A simple caption by: makerbot.com}
\label{overflow}
\end{figure}

However, let us talk first about the 3D scanner and how it can function, to better understand the idea before we dive into the dream user interface design itself, which this 3D scanner uses to perform its tasks.
First, the 3D scanner I am thinking of is actually within the screen itself.  For example, think of your laptop screen as it's able to fire some laser when it asked to do so by a command that is in the software drop down menu, or a shortcut similar to "command + P" for printing. Then, the object placed in front of it, let us say on the user's hand gets scanned in the form of a 3D model to your computer.  Also, when 3D scanner is used within the most used device by the user, it makes the user more comfortable and only needs simple instructions on how to use this kind of scanner, which is the point where it clearly leads to my dream design interface that needs to be designed in any computer, tablet, or smartphone, to allow the user to have a better interaction with the 3D scanner.

My dream interface design is a software that can be installed in any computer, tablet, or a smart-phone that uses a modern system.  The software I am dreaming of in accomplishing such tasks, starts off by introducing the user on how the software runs and what each button can do.  That is done by either showing a step by step introductory, or by showing a video the user can go through to see how this interface works.  

Moreover, the interface should feel and look like a real scanner.  How?  Well, let us first imagine how installing a new software into a system looks like. The installing process is shown exactly on the screen while the system is working on installing it, and that is done by showing a Live window that contains information that tells the user at what step the system is, and how many minutes are left for the installing process to be done.   However, my dream interface should do the same, but by showing the user exactly what is being scanned and what is not during the process of scanning and that will allow the user to accomplish the task with a responsive system showing what needs to be done.  

More, and since we haven't talked about how should the object in hand be placed in front of the screen, which we will cover in the next section.  Let me just introduce a simple idea of how the interface should be designed to communicate with the user while doing the task.  First, when launching the application, the user is supposed to know where to start since this interface introduced the first steps on where to start.  However, my dream interface design should show as we stated above a responsive messages to the user while doing the task.  One of the ways this is done, is by first showing a red point shown on the middle of the screen that detect where the object is placed. This red point should be obvious and centered, also, it should show a message to the user to tell him/her that you need to move the object in hand around the screen until this red point turns to green, where you are now set to scan your object.  

More on the design, the interface should show the ports that have been scanned and the parts that haven't been scanned by showing on the right of the screen subsections of the object that's been scanned what is not.  Therefore, this will allow the user to know exactly what section of the object needs to be placed to be scanned.

Furthermore, the interface should show and produce a sound when the object scanning has been completed.  Then, there comes the part where this interface brome more interesting.  This part after scanning an object the interface start a new page that has tools section on the left, frames of the object on the right, and more functionality that allow the user to start editing the scanned object. More on this in the next section.

Finally, we said that the design of the interface should be fully responsive, and should be demonstrative on how each task is suppose to be done.  Also we said that it should feel real by showing an exact section and demonstration of the Live process.  We also said it should then take the user to an edit page where the user is able to edit everything in the object that's been scanned.  However, the layout of the interface should be as follows, a menu bar on the top that has most of the menu bars features we know now.  Also, a start button that is bellow the menu bar which should be large an obvious for the users to see.  Also, in the middle of screen where the center red point is, there should be a large space to show Live images of the computer camera when the start button is clicked.  For example the captured images should be the background of the middle section of the software screen and the red point should be above it.  Here is a demonstration of what I mean. In the following image the square edges change color when the bar is placed correctly. However, in my dream interface is a center point that changes color from red to green when centered correctly.

\begin{figure}[ht!]
\centering
\includegraphics[width=150mm]{../../../Downloads/0_ShopSavvyHiResScanningAvatar.jpg}
\caption{A simple caption by the application "ShopSavvy" for the iPhone}
\label{overflow}
\end{figure}

 \section{Usage Scenarios}
When it comes to the usage scenarios of this interface, there are many and we mentioned some of them in the previous section.  However, let us go now into some scenarios that can be thought of when using this interface.  The first usage scenarios we will assume that the user is using this interface for the first time.  The user will click on the software icon on his/her device, and the application will launch with an introduction of what this program is capable of doing what it is not.  Then, let us say the user goes through all these instructions and he is ready to go to scan an object, let us say an espresso cup.  The user will hold the in one hand, and click on the start scan button that we mentioned.  Then the software will run the camera and show its capturing Live on the screen.  In this case, the user doesn't know where to place the object, therefore, the interface will guide the user by showing messages on the screen to direct the user of where to place. Then let's say the user saw that by placing the cup on the red point "circle," its color changed to green, then the user immediately will now that this is what he is suppose to do.  Then after one or two seconds of holding it the interface will show a count down of 3 seconds to till the start of the process. Meanwhile, the user might decide to change the position of the cup, if he do so, the countdown will reset and send a message to reposition again. The user will know he did something wrong and he committed an error. Therefore, the user will reposition the cup to the green state. 

Again, the interface as it's designed will show a countdown again, once the countdown is over, scanning starts.  Once the scanning starts as we mentioned in the design section the interface will show the sections of the cup that has been scanned on the right of the screen.  Then the software will guide the user to rotate the cup, by showing a message to the user telling him to scan the sections that haven't been scanned yet.  Therefore, the user will rotate the object and complete the task. 

Finally, the user interface will ask the user immediately after finishing the task to either go edit, or save the scanned object as images of some sort of format, or a special format for this particular interface that has the whole 3D object in one file. Then, let us say the user decided to go and click on the edit button and get directed to the new window that has tools such as the once used in photoshop to edit the 3D object.  From there the user can change the object colors and shape, while the interface is interacting with the user by showing pop-ups menus to some of the recommended tools and their way of usage.  Therefore, this will decrease the amount of errors made, and help the user to learn the interface more quickly than without those pop-up menus.  

\section{Rationale}

For this section we will discuss the interaction design in its many aspects that are used in this dream interface. The first thing about a good interface is that it should use Menus, Forms, and Dialogs to represent its content that meets the user's mental model. However, my dream user interface design should use Single Menus, Pull-Down Menus, and Popup Menus to show applicable options and instructions on how to accomplish the task. 

Furthermore, when designing my dream interface it must count for the interaction design guideline, and principle when the interface is built. The first thing that should be taken for consideration when displaying the data of this interface, the ideas of "Smith and Mosier," where they stated to be consistent in labeling and graphic conventions, as well as, standardizing abbreviations, and these guidelines should be included in the design my the menus mentioned above.  Moreover, they said to use consistent formatting in all displays, and present data only if they assist the operator, which was clearly mentioned in the above sections.  For example, pop-up menus to hint the usage of tools and button.  Also, for the second point, we mentioned how the interface presents data for user assistance constantly while accomplishing the task.

Moreover, when we talk about data entry, "Smith and Mosier," pointed out that bad data entry results in serious consequences, therefore the interface should use minimal input actions by the user, and that is exactly what my dream interface is doing. it minimizes the data input by the user by allowing the computer to scan the object, therefore, the only thing the user should do is hold the object correctly and rotate it. 

In the following we will be mentioning a good interface design principles, and show how my dream interfaces should accommodate them.  First of all, there are golden rules for a good interface, presented by "Shneiderman."  The first one we should mention is "offer informative feedback," and that is clearly what my dream interface does when the user is using it.  My dream interface provides informative feedback, and instructions as pop-up menus or as an introduction when first launching the application.  Also, when the user is holding the object the interface shows feedback by changing the color of the point where the object should be centered at. 

Furthermore, from "prevent errors," as for sure an essential part of designing a good interface.  However, in my dream interface, the interface tries to prevent errors by showing guidance to the user beforehand, and while the user is working on the task.  Finally, from "Tognazzini?s take: Sixteen first principles," my dream interfaces uses the principle of protecting the user?s work, by allowing the user to save his/her work in different forms and formats, and also, by minimizing the amount of errors committed.

\section{Usability Metrics}

For this section we will assume that the dream interface is being implemented and tested, then what are its strongest Usability Metrics presented in terms of Nielsen?s vocabulary?  Well, I believe once this interface is being tested the strongest metric will be efficiency as mentioned above.  The reason why, is that once the user is presented to the interface, the dream interface will direct the user on how to do the task.  However, that is not it, but rather when the user has already accomplished scanning an object more that ones before, it's time to see how efficient this interface is.  Furthermore, I believe my dream interface should be fairly easy when doing the task for the second time, because, you only need to open the software, click on the start button and place the object in the center and rotate when directed.  Therefore, the speed of performance should be fast.

Furthermore, one of the strongest metrics is of course, "learnability" because the interface is designed based on the previously mentioned guideline and principle of interaction design, and also uses drop-down menus and pop-up menus to guide the user towards completing the task, and finally it uses direct messages to show the user where and how to place the object.  Therefore,  this will allow the user to accomplish the basic tasks the first time they encounter the my dream interface design.

Finally, this system can be weak in some of the metrics, and I am thinking it should be error making.  The reason is that this kind of interface needs to capture precise data, and that requires precise movement, and placement to the object in front of the screen.  Therefore, users may place the object too close or too far from the scanning area "let's say above the keyboard or 3 inches away from the screen if it were a tablet."  Also, some will definitely if not all, not going to rotate the object of not instructed to do so. Whereas, some will rotate the object but not in all angles.  

\section{Target Systems}

Before, we sum up all this dream interface, we should mention what kind of system this interface can work on.  I believe this interface needs to be implemented in any photo, or movie editor as a choice from a drawdown menu or any sort of functional button.  Also, there should be a system such as photoshop, or let's call it 3D shop, that you need to run and simultaneously work with you as you scan the object.  Also, this 3D shop should allow the user to adjust the scanned object, either by changing its color, shape, and everything. Moreover, like we mentioned above, this system should be linked online to the manufacturing companies or any sort of online community that focuses on editing 3D objects, and it should allow the user to send feedback and edits to those companies.  Also, this system should be downloadable in any computer, phone, tablet or gaming device, such that it allow the user to use his/her models anywhere they go. 
% JD: This paragraph continues to mix up "utility" and "usability."  Here is another way to
%     state it: you should focus on *how the user communicates with the machine* to perform
%     a particular task, *and not the task itself*.
\section{Conclusion}

Finally, I believe this dream interface, can become a reality in the near future as there are so many research companies that tries to advance the current 3D scanners into smaller and more compatible ones.  Furthermore, the need for this kind of interface will accelerate the wheel towards making it possible in the near future. Finally, due to its importance and efficiency in terms of its usage, this interface will be the next big thing in education, art, movie making, video games playing, and manufacturing in general.
% JD: Note again that this is focused on what the system lets the user do, and now how the
%     user tells the system to do it.
\end{document}  

% JD: No citations?  For technology that is this speculative, I would expect that you have
%     at least one reference.  Otherwise, this is just science fiction.
