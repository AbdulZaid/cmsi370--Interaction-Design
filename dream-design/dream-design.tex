\documentclass[12pt, oneside]{article}   	% use "amsart" instead of "article" for AMSLaTeX format
\usepackage{geometry}                		% See geometry.pdf to learn the layout options. There are lots.
\geometry{letterpaper}                   		% ... or a4paper or a5paper or ... 
\usepackage{setspace}
%\geometry{landscape}                		% Activate for for rotated page geometry
%\usepackage[parfill]{parskip}    		% Activate to begin paragraphs with an empty line rather than an indent
\usepackage{graphicx}				% Use pdf, png, jpg, or eps§ with pdflatex; use eps in DVI mode
								% TeX will automatically convert eps --> pdf in pdflatex	
	
\usepackage{amssymb}

\title{Dream Design}
\author{Abdul Alzaid}
%\date{}							% Activate to display a given date or no date

\begin{document}
\maketitle
%\section{}
%\subsection{}
\section{Introduction}
\doublespacing
\paragraph{}
Interfaces and their designs are considered necessity in our daily technological use of modules that can be applied on a hardware, or a software.  Moreover, the interface between a user and a machine, has to be built based on some guidelines, principles, and theories of interface designs, that allow the user to have a better experience when using those interfaces.  However, in this paper I will mention my dream interface that can be applied as a software in a specifically designed machine, which therefore has to follow certain guidelines and principles of design that are going to be mentioned. So what is my dream interface?
\paragraph{}
Since we are all surrounded by so many hard objects in our lives, and those objects sometimes are needed to be shown exactly as they are on a computer or a tablet so they can be illustrated or modified in a way or another. Therefore, what I have always dreamed of is a 3D scanner from your own computer, tablet, or phone's screen. The screen can work as a scanner which scans any object you place it in front of it, to an exact 3D model of it in the software that you're using. Moreover, the screen will capture objects in full color with multi-laser precision. The laser comes out from the screen and scans the object and deliver a 3D model of that object to your computer. 
\section{Design}
\paragraph{}
The 3D scanner interface design should be implemented in every device that has a screen in it.  For example, PCs, tablets, Televisions, smart-phones, and video game devices. Furthermore, the interface design should be efficient, and easy to use by all kind of users. Therefore, it should make the user interaction with the machine easier and more convenient than ever in accomplishing this kind of tasks. 
\paragraph{}
Furthermore, we all know that there are some 3D printers out there, and few 3D scanners as well.  However, the 3D scanners that are available now are very expensive and they are in the form of a different device that is connected to a computer and require a whole workstation to perform their task.  Also, the ones available now needs more than one positioning to the object as shown in the next image

\begin{figure}[ht!]
\centering
\includegraphics[width=90mm]{../../Downloads/photo41.jpg}
\caption{A simple caption by: makerbot.com}
\label{overflow}
\end{figure}

\paragraph{}
  However, the one I am dreaming of is actually within the screen itself.  For example, think of your laptop screen as it's able to fire some laser when it asked to do so by a command that is in the software drop down menu, or a shortcut similar to "command + P" for printing. Then, the object placed in front of it, let us say on the user's hand gets scanned in the form of a 3D model to your computer. Also, when 3D scanner is used within the most used device by the user, it makes the user more comfortable and only needs simple instructions on how to use this kind of interface, which is clearly a very useful interface as we will show the usage scenarios in the next section.
\section{Usage Scenarios}
\paragraph{}
When it comes to the usage scenarios of this interface, there are many.  For instance, making movies requires implementing real objects into them. Let us say you are a 3D Modeler that needs to design some kind of buildings, cars, weapons, and you only have the smaller version of that object on your hand. Usually visual artist will place that object and draw it using some kind of application in their computers. Of course, this takes time with less precision.  But, with this kind of interface the only thing you need to do is hold that object in front of the computer's screen and it will laser scan the object with hundred percent precision.  
\paragraph{}
Another usage, we all have objects in hand, and sometimes we complain about their design, layout, and maybe functionality. For example, you're holding a glasses or sunglasses, and that glasses doesn't fit very well, or maybe it can break easily and you have some adjustment in mind. So, what you do is scan it and then edit it.  When you're done, send a copy of your work to the manufacturer and they will get you what you asked for.  Moreover, everyone has things in mind and don't know how or where to suggest their ideas.  So, this kind of interface if used with a specified application that allow the user to edit the 3D model easily, and link the application to any company that manufactures the product, then I beleive this will revolutionize the industry with so many innovations and adjustments to the current objects we have.  

\section{Rationale}
\paragraph{}
\section{Usability Metric}
\paragraph{}
\section{Target Systems}
\paragraph{}
\section{Conclusion}
\paragraph{}

\end{document}  